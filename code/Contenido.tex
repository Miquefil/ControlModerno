% \tableofcontents
% \newpage

%Definiciónes de comandos y snippets

%SNIPPETS IMPORTANTES:
% gr: pone figuras al 75% de escala
% eq: pone ecuación de linea $$
% mat: matriz de corchetes cuadrados 3x3
% fd: math inline
% ds: display math
%%%%%%%%%%%%%%%%%%%%%%%%%%%%%%%%%%%%%%%
%Comienzo de contenido

\section{Matrices}
\subsection{Operación determinante}
\begin{definition}
    \textbf{Método de los Cofactores:} Sea $A$ una matriz de dimensiones $n \cdot n$, 
    podemos entonces calcular el determinante de $A$ ($\determinant(A)$ o $|A|$) como sigue.
    Definimos el cofactor de fila $i$ y columna $j$ como:
    \begin{equation*}
        A_{ij} = (-1)^{i+j} \; |M_{ij}|,
    \end{equation*}
    donde $|M_{ij}|$ es el determinante de la matriz $M_{ij}$, que es la matriz de orden $(n-1)\cdot(n-1)$ resultante
    de quitar la i-ésima fila y j-ésima columna de $A$. Luego el determinante de A es la suma de los productos de los
    elemntos de una fila o columna con sus cofactores, es decir.
    \begin{equation*}
        |A| = a_{k1} \cdot A_{k1} + \dots + a_{kn} \cdot A_{kn},
    \end{equation*}
    o también,
    \begin{equation*}
        |A| = a_{1k} \cdot A_{1k} + \dots + a_{nk}\cdot A_{nk}, \; \forall k \in {1, \dots, n}.
    \end{equation*}
\end{definition}

\begin{example}[Filas o columnas de ceros]
    El cálculo de
    \begin{equation*}
        |A| = 
        \begin{vmatrix}
            \sigma & \omega & 0 \\
            \omega & \sigma & 0 \\
            0 & 0 & \lambda \\
        \end{vmatrix},
    \end{equation*}
    se puede realizar rápidamente por método de cofactores, realizando el método en la tercer fila o columna, quedando:
    \begin{equation*}
        |A| = (-1)^{3+3}\lambda \cdot \begin{vmatrix}
            \sigma & \omega \\
            \omega & \sigma \\
        \end{vmatrix} = \lambda \; (\sigma^2 - \omega^2).
    \end{equation*}
\end{example}

\textbf{Propiedades:} Sean A y B matrices cuadradas, entonces

\begin{bangenumerate}
    \item $|A| = |A^t|$,
    \item $|A \cdot B| = |A|\cdot |B|$,
    \item $|A^{-1}| = |A|^{-1}$,
    \item Si $A$ es triangular o diagonal entonces $|A|$ es igual al producto de
    su diagonal principal.
\end{bangenumerate}

\subsection{Operación inversa}
\begin{definition}
    Si $A$ es una matriz no nula de dimensión $n \cdot n$ entonces
    \begin{equation*}
        A^{-1}  = \frac{Adj(A)^{T}}{|A|},
    \end{equation*}
    donde $Adj(A)$ es la matriz adjunta, es decir aquella que está compuesta por los cofactores de A.
\end{definition}

\textbf{Propiedades:}
\begin{bangenumerate}
    \item $(A^{-1})^{-1} = A$
    \item! $(A\cdot B)^{-1} = B^{-1}\cdot A^{-1}$
    \item $(A^T)^{-1} = (A^{-1})^T$
\end{bangenumerate}

\begin{theorem}
    Una matriz tiene inversa si y sólo si su determinante es distinto
    de cero
    \begin{equation*}
        \exists A^{-1} \Leftrightarrow |A| \neq 0
    \end{equation*}
\end{theorem}


\subsection{Operación Trasposición}
La trasposición de una matriz ($n \cdot m$), rectangular o cuadrada, es la
reflexión de los elementos respecto de su diagonal principal, adquiriendo
la forma ($m \cdot n$).
\begin{equation*}
    \begin{bmatrix}
        a & b & c \\
        d & e & f \\
    \end{bmatrix}^{T}  =
    \begin{bmatrix}
        a & b \\
        c & d \\
        e & f \\
    \end{bmatrix}
\end{equation*}

\textbf{Propiedades:}
\begin{bangenumerate}
    \item Involutiva: $(A^t)^t = A$
    \item Distributiva: $(A+B)^t = A^t + B^t$
    \item! Producto: $(A \cdot B)^t = B^t \cdot A^t$
\end{bangenumerate}


\newpage
\section{Espacios Vectoriales y Transformaciones Lineales}
